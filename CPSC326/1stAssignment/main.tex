\documentclass[12pt]{article}
\usepackage{amsmath}
\usepackage{amssymb}
\usepackage[top=1in, bottom=1in, left=1in, right=1in]{geometry}
\usepackage{titling}
\setlength{\droptitle}{-3em}

\begin{document}

\title{Lambda Calculus Constructions}
\author{
Greyson Knippel \\[4pt]
CPSC 326-01 --- Gonzaga University, compiled with \LaTeX \\[4pt]
Professor Johnson
}
\date{\today}
\maketitle

\section*{to start off:}

\[
\text{SUCC } X = X + 1
\]

\[
\text{PRED } X = X - 1
\]

\[
(= X\, Y) \Rightarrow \text{T/F}
\]

\section*{construction of ADD}

using recursion: \(R\), ADD can be defined as:

\[
\begin{aligned}
\text{R ADD } X\, Y &= R\, \text{ADD}\, X\, Y \\
&= \text{(if) } (= Y\, 0)\, \text{(then) } X\, \text{(else) }(\text{ADD } (\text{SUCC } X)\, (\text{PRED } Y))
\end{aligned}
\]

\section*{construction of MULT}

MULT can also be defined by using recursion R: but this time its simply just repeated addition.

\[
\begin{aligned}
\text{R MULT } X\, Y &= R\, \text{MULT}\, X\, Y \\
&= \text{(if) } (= Y\, 0)\, \text{(then) } 0\, \text{(else) }(\text{ADD } X\, (\text{MULT } X\, (\text{PRED } Y)))
\end{aligned}
\]

\section*{example --- ADD 2 3}

\[
\begin{aligned}
\text{R ADD } 2\, 3 &= \text{(if) } (= 3\, 0)\, 2\, (\text{ADD } (\text{SUCC } 2)\, (\text{PRED } 3)) \\
&= \text{(if) } (= 2\, 0)\, 3\, (\text{ADD } (\text{SUCC } 3)\, (\text{PRED } 2)) \\
&= \text{(if) } (= 1\, 0)\, 4\, (\text{ADD } (\text{SUCC } 4)\, (\text{PRED } 1)) \\
&= \text{(if) } (= 0\, 0)\, 5\, (\text{ADD } (\text{SUCC } 5)\, (\text{PRED } 0)) \\[0.5em]
&\text{- basically (= y 0) is your base case,} \\
&\quad \text{then use recursion with SUCC AND PRED.} \\[0.5em]
&= 5
\end{aligned}
\]

\section*{example --- MULT 2 4}

\[
\begin{aligned}
\text{MULT } 2\, 4 &= \text{(if) } (= 4\, 0)\, \text{(then) } 0\, \text{(else) }(\text{ADD } 2\, (\text{MULT } 2\, (\text{PRED } 4))) \\
&= \text{(if) } (= 3\, 0)\, \text{(then) } 0\, \text{(else) }(\text{ADD } 2\, (\text{MULT } 2\, (\text{PRED } 3))) \\
&= \text{(if) } (= 2\, 0)\, \text{(then) } 0\, \text{(else) }(\text{ADD } 2\, (\text{MULT } 2\, (\text{PRED } 2))) \\
&= \text{(if) } (= 1\, 0)\, \text{(then) } 0\, \text{(else) }(\text{ADD } 2\, (\text{MULT } 2\, (\text{PRED } 1))) \\
&= \text{(if) } (= 0\, 0)\, \text{(then) } 0\, \text{(else) }(\text{ADD } 2\, (\text{MULT } 2\, (\text{PRED } 0))) \\[0.5em]
&\text{and adding all the recursions together with the ADD block\ldots} \\
&\text{- basically we are just using ADD's recursion,} \\
&\quad \text{doing repeated addition with MULT's recursion.} \\[0.5em]
&= 8
\end{aligned}
\]

\end{document}