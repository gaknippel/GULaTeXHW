\documentclass[12pt]{article}
\usepackage[margin=1in]{geometry}
\usepackage{listings}
\usepackage{xcolor}
\usepackage{amsmath}
\usepackage{amssymb}
\usepackage{titling}

\lstdefinelanguage{Lean}{
  keywords={structure, where, def, Float, if, then, else},
  keywordstyle=\color{blue}\bfseries,
  ndkeywords={RectangularPrism, Segment, Point3D},
  ndkeywordstyle=\color{purple},
  identifierstyle=\color{black},
  sensitive=true,
  comment=[l]{--},
  commentstyle=\color{gray}\ttfamily,
  stringstyle=\color{red}\ttfamily,
  morestring=[b]',
  morestring=[b]"
}

\lstset{
  language=Lean,
  basicstyle=\ttfamily\small,
  numbers=left,
  numberstyle=\tiny\color{gray},
  stepnumber=1,
  numbersep=8pt,
  showstringspaces=false,
  breaklines=true,
  frame=single,
  backgroundcolor=\color{white},
  tabsize=2,
  captionpos=b
}

\begin{document}

\title{Lean Structures Exercises}
\author{
Greyson Knippel \\[4pt]
CPSC 326-01 --- Gonzaga University, compiled with \LaTeX \\[4pt]
Professor Johnson
}
\date{\today}
\maketitle


\section*{problem 1}

\subsection*{part (a): }

\begin{lstlisting}
structure RectangularPrism where
  height : Float
  width : Float
  depth : Float
\end{lstlisting}

\subsection*{Part (b): }

\begin{lstlisting}
def volume (r1 : RectangularPrism) : Float :=
  r1.height * r1.width * r1.depth

def myPrism : RectangularPrism := 
  {height := 5.0, width := 2.0, depth := 6.0}

#eval volume myPrism
\end{lstlisting}

\subsection*{Part (c):}

\begin{lstlisting}
structure Segment where
  s1 : Float
  s2 : Float
  f1 : Float
  f2 : Float

def mySegment : Segment := 
  {s1 := 0.0, s2 := 0.0, f1 := 5.0, f2 := 4.0}

def length (m : Segment) : Float :=
  Float.sqrt (((m.f1 - m.s1)^2) + ((m.f2 - m.s2)^2))

#eval length mySegment
\end{lstlisting}

\section*{problem 2:}

\begin{enumerate}
  \item \textbf{strict:} functions and parameters are fully evaluated before the function body begins evaluation.
  
  \item \textbf{pure:} programs cannot have side effects.
  
  \item \textbf{functional:} functions are primary objects of interest; computation is evaluating functions (mathematical expressions).
  
  \item \textbf{has dependent types:} types can contain programs that compute types.
\end{enumerate}

\section*{problem 3:}

\begin{lstlisting}
structure Point3D where
  x : Float
  y : Float
  z : Float
\end{lstlisting}

\section*{problem 4:}

\begin{lstlisting}
def minimumComponent (p : Point3D) : Float :=
  if (p.x < p.y) then
    if p.x < p.z then p.x else p.z
  else
    if p.y < p.z then p.y else p.z
\end{lstlisting}

\section*{problem 5:}

\begin{lstlisting}
def midpoint (p1 : Point3D) (p2 : Point3D) : Point3D :=
  { x := (p1.x - p2.x) /2.0,
    y := (p1.y - p2.y) /2.0,
    z := (p1.z - p2.z) /2.0 }
\end{lstlisting}

\end{document}